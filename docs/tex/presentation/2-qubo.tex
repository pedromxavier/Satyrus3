\documentclass[brazil, MathSerif, aspectratio = 169]{beamer}


\usetheme{OsloMet}
\usepackage{homework}
\usepackage{style}
\usepackage{bm}

\author[Pedro]
{Pedro Maciel Xavier}
\title{II - QUBO}
\subtitle{\small Programação não-linear \textbullet{} Otimização Inteira}

\begin{document}
% Use
%
%     \begin{frame}[allowframebreaks]
%
% if the TOC does not fit one frame.
\begin{frame}%%[allowframebreaks]%%
    {Sumário}
    \tableofcontents
\end{frame}

%% Disable the logo in the lower right corner:
\hidelogo

\section{\emph{Ising Model}}

\begin{frame}{\emph{Ising Model}}
    \begin{definition}[\emph{Quadratic Unconstrained Binary Optimization}]
        Um problema de otimização é assim denominado se pode ser escrito na forma 
        \begin{align*}
            \mathbb{H} = \sum_{i} \vet{h}_{i} \ket{s_i} + \sum_{i < j} \vet{J}_{i, j} \ket{s_{i}} \ket{s_{j}}
        \end{align*}
        onde $f(\vet{x}) = \vet{x}\T \vet{Q}\, \vet{x}$ para $\vet{Q} \in \R^{n \times n}$. Mais especificamente, $\vet{Q}$ é uma matriz simétrica ou triangular superior.
    \end{definition}  
\end{frame}

\section{\emph{Quadratic Unconstrained Binary Optimization}}

\subsection{Definição e características}

\begin{frame}{QUBO}
    \begin{definition}[\emph{Quadratic Unconstrained Binary Optimization}]
        Um problema de otimização é assim denominado se pode ser escrito na forma 
        \begin{align*}
            \text{minimizar } &f(\vet{x})\\
            \text{sujeito a } &\vet{x} \in \set{0, 1}^{n}
        \end{align*}
        onde $f(\vet{x}) = \vet{x}\T \vet{Q}\, \vet{x}$ para $\vet{Q} \in \R^{n \times n}$. Mais especificamente, $\vet{Q}$ é uma matriz simétrica ou triangular superior.
    \end{definition}    
\end{frame}

%% ---------------------------------------------------------------------------

\subsection{Resolvendo}

\begin{frame}{Annealing}
    O \emph{Annealing} (ou Têmpera) é um dos processos mais populares para solucionar o \emph{QUBO}. Dentre os principais métodos desta classe estão:
    \begin{itemize}[<+-|structure@+>]
        \item \emph{Quantum Annealing}
        \item \emph{Digital Annealing}
        \item \emph{Simulated Annealing}
    \end{itemize}
\end{frame}

%% Enable the logo in the lower right corner:
\section{Satyrus}

\begin{frame}{Satyrus}

    No contexto do Satyrus, temos uma equação de energia a minimizar dada por
        \begin{align*}
            \mathbb{E} &= \mathbb{E}_\text{opt} + \mathbb{E}_\text{int}\\
                       &= \sum_{i} \mathcal{H} \left( \varphi_i \right) %%
                       + \sum_j \pmb{\lambda}_j \mathcal{H} \left( \neg \varphi_j \right)
        \end{align*}
    onde $\lambda_j$ é a penalidade associada à $j$-ésima restrição de integridade e $\mathcal{H} \left(\,\cdot\,\right)$ é o mapeamento. Portanto, é possível escrever um problema modelado pelo Satyrus como
        \begin{align*}
            \text{minimizar } &f(\mathbf{x}) + \pmb{\lambda} \cdot g(\mathbf{x})\\
            \text{sujeito a } &\mathbf{x} \in \{0, 1\}^{n}
        \end{align*}
\end{frame}

\begin{frame}{Satyrus}%%
    Resta saber se é possível escrever
        $f(\vet{x}) + \pmb{\lambda} \cdot g(\vet{x}) = \vet{x}\T \vet{Q} \vet{x}$
    
    
\end{frame}

\section{Próximos Tópicos}
\begin{frame}{Próximos Tópicos}

    Compreender melhor:
    \begin{enumerate}%%[<+-|alert@+>]
        \item O princípio da dualidade (e suas demonstrações).

        \item A relação entre o lagrangeano de um sistema e seu hamiltoniano.
        
        \item Modelagem QUBO
    \end{enumerate}
\end{frame}

\section{Referências}
\begin{frame}%%[allowframebreaks]%%
    {Referências}

    \begin{thebibliography}{}

        % Article is the default.
        \setbeamertemplate{bibliography item}[article]

        \bibitem{brezhneva:2011}
        Olga Brezhneva, Alexey A. Tret’yakov, Stephen E. Wright
        \newblock \emph{A short elementary proof of the Lagrange multiplier theorem}
        \newblock \emph{Optimization Letters}.
        \newblock Springer-Verlag, 2011.

        \setbeamertemplate{bibliography item}[book]

        \bibitem{luenberger:2008}
        David G. Luenberger, Yiniu Ye
        \newblock \emph{Linear and Nonlinear Programming}
        \newblock Springer-Verlag, 2008.

        \setbeamertemplate{bibliography item}[triangle]

    \end{thebibliography}
\end{frame}

\end{document}
