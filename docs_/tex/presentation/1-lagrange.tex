\documentclass[brazil, MathSerif, aspectratio = 169]{beamer}


\usetheme{OsloMet}
\usepackage{style}
\usepackage{bm}

\author[Pedro]
{Pedro Maciel Xavier}
\title{I - Otimização Lagrangeana}
\subtitle{\small Programação não-linear \textbullet{} Otimização com restrições}

\begin{document}

\section{Sumário}
% Use
%
%     \begin{frame}[allowframebreaks]
%
% if the TOC does not fit one frame.
\begin{frame}[allowframebreaks]{Table of contents}
    \tableofcontents
\end{frame}

\section{Multiplicadores de Lagrange}

\subsection{Teorema}

%% Disable the logo in the lower right corner:
\hidelogo

\begin{frame}{Multiplicadores de Lagrange}

    \begin{theorem}[Multiplicadores de Lagrange]
        Sejam $f: \mathbb{R}^{m} \to \mathbb{R}$ e $g: \mathbb{R}^{m} \to \mathbb{R}^n$ de classe $\mathcal{C}^{1}$ onde $\mathbf{x}^{\circ} \in \mathbb{R}^{m}$ minimiza (maximiza) localmente o sistema
        $$ \min_{\mathbf{x} \in \mathbb{R}^{m}}\left\{ f(\mathbf{x}) \middle| g(\mathbf{x})  = \alpha \right\}, \alpha \in \mathbb{R}^{n} $$
        Se a matriz Jacobiana $n \times m$ $g'(\mathbf{x}^{\circ})$ tiver linhas linearmente independentes, então existe um único $\pmb{\lambda}^{\circ} \in \mathbb{R}^{n}$ que satisfaz
        $$ f'(\mathbf{x}^{\circ}) = \pmb{\lambda}^{\circ} g'(\mathbf{x}^{\circ}) $$
    \end{theorem}

    
    Demonstrações são encontradas em \cite{brezhneva:2011} e \cite{luenberger:2008}.
    
        
\end{frame}

%% Enable the logo in the lower right corner:
%%\showlogo

\subsection{Exemplo}

\begin{frame}{Multiplicadores de Lagrange}
    O teorema permite reescrever um problema definido pelas funções $f(\mathbf{x})$ e $g(\mathbf{x})$ como o sistema de $m + n$ variáveis
        $$\mathcal{L}(\mathbf{x}, \pmb{\lambda}) = \nabla f(\mathbf{x}) + \pmb{\lambda} \nabla g(\mathbf{x}) = 0 $$
    chamado de \emph{Lagrangeano} do sistema.
    \begin{example}
        Considere o seguinte problema \cite{luenberger:2008}:
        \begin{align*}
            \text{minimizar } &x_1 x_2 + x_2 x_3 + x_1 x_3 \\
            \text{sujeito a } &x_1 + x_2 + x_3 = 3
        \end{align*}


    \end{example}
\end{frame}

\begin{frame}{Multiplicadores de Lagrange}
    \begin{example}
        Reescrevemos
        \begin{align*}
            \text{minimizar } &x_1 x_2 + x_2 x_3 + x_1 x_3 \\
            \text{sujeito a } &x_1 + x_2 + x_3 = 3
        \end{align*}
        como
        $$\begin{array}[]{@{}cccc@{}}
            \phantom{x_1} &+\,\, x_2 &+\,\, x_3 &+\,\, \lambda = 0 \\
            x_1 &\phantom{+\,\, x_2} &+\,\, x_3 &+\,\, \lambda = 0 \\
            x_1 &+\,\, x_2 &\phantom{+\,\, x_3} &+\,\, \lambda = 0
        \end{array}$$
        cuja solução é $x_1 = x_2 = x_3 = 1$ e $\lambda = -2$.
    \end{example}
\end{frame}

\section{Relaxação Lagrangeana}

\begin{frame}{Relaxação Lagrangeana}
    \begin{definition}[Relaxação Lagrangeana]
        Dado um problema na forma
        \begin{align*}
            \text{minimizar } &\mathbf{c} \cdot \mathbf{x}\\
            \text{sujeito a } & \mathbf{A} \mathbf{x} = \mathbf{b}\\
                &\mathbf{x} \in X
        \end{align*}
        Podemos embutir parte das restrições na função objetivo 
        \begin{align*}
            \text{minimizar } &\mathbf{c} \cdot \mathbf{x} + \pmb{\lambda} \cdot \left( \mathbf{A} \mathbf{x} - \mathbf{b} \right)\\
            \text{sujeito a } &\mathbf{x} \in X
        \end{align*}
        penalizando a violação das restrições em $\mathbf{A} \mathbf{x} = \mathbf{b}$.
    \end{definition}
\end{frame}

\begin{frame}{Relaxação Lagrangeana}
    \begin{definition}[Relaxação Lagrangeana]
        Nos referimos a este problema como Relaxação Langrangeana ou Dual Lagrangeano do problema original e chamamos a função
            $$\mathcal{L}(\mathbf{x}, \pmb{\lambda}) = \min \left\{%%
                \mathbf{c} \cdot \mathbf{x} + \pmb{\lambda} \cdot \left(%%
                \mathbf{A} \mathbf{x} - \mathbf{b}
                \right)\, \middle|\, \mathbf{x} \in X
            \right\}$$
        de função Lagrangeana.
    \end{definition}
\end{frame}

\section{Satyrus}

\begin{frame}{Satyrus}

    No contexto do Satyrus, temos uma equação de energia a minimizar dada por
        \begin{align*}
            \mathbb{E} &= \mathbb{E}_\text{opt} + \mathbb{E}_\text{int}\\
                       &= \sum_{i} \mathcal{H} \left( \varphi_i \right) %%
                       + \sum_j \pmb{\lambda}_j \mathcal{H} \left( \neg \varphi_j \right)
        \end{align*}
    onde $\lambda_j$ é a penalidade associada à $j$-ésima restrição de integridade e $\mathcal{H} \left(\,\cdot\,\right)$ é o mapeamento. É possível escrever um problema modelado no Satyrus como
        \begin{align*}
            \text{minimizar } &f(\mathbf{x}) + \pmb{\lambda} \cdot g(\mathbf{x})\\
            \text{sujeito a } &\mathbf{x} \in \{0, 1\}^{n}
        \end{align*}
\end{frame}

\section{Próximos Tópicos}
\begin{frame}{Próximos Tópicos}

    Compreender melhor:
    \begin{enumerate}%%[<+-|alert@+>]
        \item O princípio da dualidade (e suas demonstrações).

        \item A relação entre o lagrangeano de um sistema e seu hamiltoniano.
        
        \item Modelagem QUBO
    \end{enumerate}
\end{frame}

\section{References}
\begin{frame}[allowframebreaks]{References}

    \begin{thebibliography}{}

        % Article is the default.
        \setbeamertemplate{bibliography item}[article]

        \bibitem{brezhneva:2011}
        Olga Brezhneva, Alexey A. Tret’yakov, Stephen E. Wright
        \newblock \emph{A short elementary proof of the Lagrange multiplier theorem}
        \newblock \emph{Optimization Letters}.
        \newblock Springer-Verlag, 2011.

        \setbeamertemplate{bibliography item}[book]

        \bibitem{luenberger:2008}
        David G. Luenberger, Yiniu Ye
        \newblock \emph{Linear and Nonlinear Programming}
        \newblock Springer-Verlag, 2008.

        \setbeamertemplate{bibliography item}[triangle]

    \end{thebibliography}
\end{frame}

\end{document}
