\documentclass[brazil, MathSerif, aspectratio = 169]{beamer}

\usetheme{OsloMet}
\usepackage{homework}
\usepackage{multicol}
\usepackage{style}
\usepackage{bm}

\author[Pedro Maciel]
{Pedro Maciel Xavier\break{\scriptsize sob orientação de Priscila Machado Vieira Lima}}
\title{Satyrus III}
\subtitle{\small }

\begin{document}
% Use
%
%     \begin{frame}[allowframebreaks]
%
% if the TOC does not fit one frame.

\begin{frame}[fragile]{Sumário}%%
    \begin{multicols}{2}
        \tableofcontents
    \end{multicols}
\end{frame}


%% Disable the logo in the lower right corner:
\hidelogo

%% Enable the logo in the lower right corner:
\section{Fundamentos}
\SectionPage

\subsection{}
\begin{frame}{}
\end{frame}

\section{O Compilador}
\SectionPage

\subsection{Mapeamento}
\newcommand{\HH}[1]{\text{H}\left(#1\right)}
\begin{frame}{Mapeamento}
    \begin{align*}
        \HH{\text{F}}   & = 0                               \\
        \HH{\text{V}}   & = 1                               \\
        \HH{\neg p}     & = 1 - \HH{p}                      \\
        \HH{p \wedge q} & = \HH{p} \HH{q}                   \\
        \HH{p \vee q}   & = \HH{p} + \HH{q} - \HH{p} \HH{q}
    \end{align*}
\end{frame}


\subsection{Penalidades}
\begin{frame}{Penalidades}
    No contexto do Satyrus, temos uma equação de energia a minimizar dada por
    \begin{align*}
        \mathbb{E} & = \mathbb{E}_\text{opt} + \mathbb{E}_\text{int} \\
                   & = \sum_{i} \HH{\varphi_i}                       %%
        + \sum_j \pmb{\lambda}_j \HH{\neg \varphi_j}
    \end{align*}
    onde $\lambda_j$ é a penalidade associada à $j$-ésima restrição de integridade e $\HH{\,\cdot\,}$ é o mapeamento. Portanto, é possível escrever um problema modelado pelo Satyrus como
    \begin{align*}
        \text{minimizar } & f(\mathbf{x}) + \pmb{\lambda} \cdot g(\mathbf{x}) \\
        \text{sujeito a } & \mathbf{x} \in \{0, 1\}^{n}
    \end{align*}
\end{frame}


\begin{frame}{Penalidades}%%
    Resta saber se é possível escrever
    $f(\vet{x}) + \pmb{\lambda} \cdot g(\vet{x}) = \vet{x}\T \vet{Q}\, \vet{x}$.
    Como, por construção, tanto $f(\vet{x})$ quanto $g(\vet{x})$ são polinômios nas componentes de $\vet{x}$, isso pode ser feito ao aplicar uma redução dos termos com três ou mais variáveis.
\end{frame}

\subsection{Restrições}
\begin{frame}[fragile]{Restrições}%%
    \begin{enumerate}
        \item Integridade
        \begin{lstlisting}[style=SATStyle, gobble=8]
        (int) constraint_X[1]:
            forall {i = [1:n]}
            exists {j = [1:n]}
            unique {k = [1:n]}
            (x[i] & y[j]) -> z[k];
        \end{lstlisting}
        \item Otimalidade
        \begin{lstlisting}[style=SATStyle, gobble=8]
        (opt) constraint_Y: exists {i = [1:n]} c[i];
        \end{lstlisting}
    \end{enumerate}

\end{frame}


\section{Exemplos de Modelagem}
\SectionPage

\subsection{Coloração de Grafos}
\begin{frame}{Coloração de Grafos}

\end{frame}

\subsection{TSP}
\begin{frame}{TSP}

\end{frame}

\subsection{Caixeiro Pintor}
\begin{frame}{Caixeiro Pintor}

\end{frame}

\section{Instruções}

\subsection{Instalação}
\begin{frame}%%
    {Instalação}

    \begin{enumerate}
        \item Através do \textit{Python Package Index} (PyPI)\\%%
              \texttt{%%
                  \$ pip install satyrus\\%%
                  \$ satyrus --help\\%%
                  \$ python -m satyrus --help
              }
        \item Por meio do código-fonte\\%%
              \texttt{%%
                  \$ git clone https://github.com/pedromxavier/Satyrus3\\%%
                  \$ cd Satyrus3\\%%
                  \$ python setup.py install
              }
    \end{enumerate}
\end{frame}

\subsection{Uso}
\begin{frame}%%
    {Uso}

    \texttt{%%
        \$ satyrus problem.sat -o \{text, csv, gurobi, dwave, ...\}%%
    }

\end{frame}

\section{Conclusão}
\begin{frame}{Conclusão}

\end{frame}

\section*{Referências}
\begin{frame}%%[allowframebreaks]%%
    {Referências}

    \begin{thebibliography}{}

        % Article is the default.
        \setbeamertemplate{bibliography item}[article]

        \bibitem{glover:2016}
        Fred Glover, Gary Kochenberger, Yu Du
        \newblock \emph{Quantum Bridge Analytics I: A Tutorial on Formulating and Using QUBO Models.}
        % \newblock \emph{Optimization Letters}.
        % \newblock Springer-Verlag, 2011.

        % \setbeamertemplate{bibliography item}[book]

        % \bibitem{luenberger:2008}
        % David G. Luenberger, Yiniu Ye
        % \newblock \emph{Linear and Nonlinear Programming}
        % \newblock Springer-Verlag, 2008.


        \setbeamertemplate{bibliography item}[online]

        \bibitem{dwave:2016}
        D-Wave Systems
        \newblock \emph{Problem-Solving Handbook}
        \newblock \url{https://docs.dwavesys.com/docs/latest/c_handbook_3.html}

        \setbeamertemplate{bibliography item}[article]

        \bibitem{boros:2002}
        Endre Boros, Peter L. Hammer
        \newblock \emph{Pseudo-Boolean optimization}
        \newblock Elsevier, Discrete Applied Mathematics, 2002.
    \end{thebibliography}
\end{frame}

\end{document}
