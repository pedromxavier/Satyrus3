\documentclass[12pt]{satyrus}
\usepackage{satyrus}
\usepackage[brazil]{babel}
\usepackage{tikz}

\tikzset{%
	>=latex, %melhor aparência para as setas
	inner sep=5pt,outer sep=0pt, %sem separação entre os nós
}

\title{Satyrus III}
\author{Pedro Maciel Xavier}

\begin{document}
    \maketitle
    
    \newpage

    \tableofcontents

    \chapter{Introdução}
    
    - SATyrus é uma plataforma\\
    - SATish é a linguagem\\
    
    \section{Motivação}

    \section{Ficha Técnica}

    \section{Implementação}

    \section{Uso}
    
    \subsection{Instalação}
    
    A instalação pode demandar privilégios de administrador.
    
	\begin{bash}
	~$\text{\$}$ git clone
	~$\text{\$}$ cd Satyrus3
	~/Satyrus3$\text{\$}$ sudo python3 setup.py install
	\end{bash}
	
	\begin{shell}
	C:\Users\User> git clone
	C:\Users\User> cd Satyrus3
	C:\Users\User\Satyrus3> python setup.py install
	\end{shell}

    \subsection{Execução}
    
    Escreva seu código em um arquivo de extensão \code{.sat}.
    
	\begin{bash}
	~$\text{\$}$ satyrus script.sat
	\end{bash}
	
	\begin{shell}
	C:\Users\User> python -m satyrus script.sat
	\end{shell}

    \chapter{Conceitos Teóricos}

	\section{Compiladores}
	
	Um compilador é um programa que transforma o código de um programa em um outro código, numa linguagem potencialmente diferente da linguagem de entrada\cite{aho:86}. O caso de uso mais comum se dá entre linguagens como \textit{C} e \textit{Fortran} que são traduzidas para o \textit{Assembly}, permitindo expressar instruções de máquina dando como entrada expressões de mais alto nível, ou seja, mais próximas da linguagem natural. Uma outra aplicação recorrente para os compiladores é a otimização de programas. Neste caso, a saída pode estar escrita na mesma linguagem que a entrada, representando o mesmo programa, mas com uma sequência de instruções mais eficiente que o original. \par
	
	\begin{figure}[H]
		\centering
		\begin{tikzpicture}[nodes={fill=gray!20},
		row sep=0.3cm,column sep=0.5cm]
			\node[rectangle] (input) at (0,0) {Programa fonte};
			\node[rectangle, fill=blue!30] (compiler) at (4, 0) {Compilador};
			\draw[->] (input) to[out=0,in=180] (compiler);
			\node[rectangle] (target) at (8, 0) {Programa alvo};
			\draw[->] (compiler) to[out=0,in=180] (target);
			\node[rectangle, fill=red!30] (error) at (4, -2) {Mensagens de erro};
			\draw[->] (compiler) to (error);
		\end{tikzpicture}
		\caption{A estrutura básica de um compilador.}
	\end{figure}

	O processo inteiro de compilação se dá em diversas fases, listadas abaixo:
	
	\begin{enumerate}
		\item Análise Léxica
		\item Análise Sintática
		\item Análise Semântica
		\item Geração de Código Intermediário
		\item Otimização de Código
		\item Geração de Código
	\end{enumerate}
	

    \section{Lógica Proposicional}
    
   	A Lógica Proposicional é um sistema formal capaz de representar elementos do universo de discurso de maneira a evitar ambiguidades. A linguagem da Lógica Proposicional é constituída pela composição dos símbolos de um determinado alfabeto, que apresentamos a seguir.
   	
   	\subsection{Alfabeto}
   	
   	O Alfabeto $\Sigma$ da Lógica Proposicional conta com símbolos para as variáveis proposicionais, constantes e operadores, além dos símbolos auxiliares (parênteses) usados para alterar a precedência dos operadores. As constantes, que representam os valores Verdadeiro e Falso aparecem em diversas grafias, mas sempre como um par de símbolos. Os usos mais comuns são $\{1, 0\}$, $\{\text{\sffamily T}, \text{\sffamily F}\}$ ou até mesmo $\{\top, \bot\}$. Os operadores usuais são $\neg$, $\vee$, $\wedge$, $\rightarrow$ com extensão para o uso de $\leftarrow$ e $\leftrightarrow$. As variáveis são, em geral, denotadas por letras do alfabeto latino\footnote{Em alguns textos, principalmente em Lógica Matemática\cite{logic}\cite{logic}, vemos o uso de letras maiúsculas para representar as variáveis proposicionais, como $P$, $Q$ e $R$. Já em publicações que tratam de Lógica Computacional\cite{logic}\cite{logic} é comum que sejam usadas letras minúsculas nesse caso, como $x$, $y$, e $z$. Neste trabalho vamos nos ater a esta última variante.}.
   		$$\Sigma = \{1, 0, \neg, \vee, \wedge, \rightarrow, \leftarrow, \leftrightarrow, x, y, z, ...\}$$
   	Tendo o alfabeto, dizemos que $\Sigma^{\ast}$ é o conjunto de todas as palavras que podem ser formadas a partir de $\Sigma$. Vejamos uma definição formal. Seja $\Sigma^{k}$ o conjunto das palavras de tamanho $k$ formadas pela concatenação de símbolos em $\Sigma$, temos
   		\begin{align*}
   		\Sigma^{0} =& \{\square\}\\
   		\Sigma^{1} =& \Sigma\\
   		 \vdots~&\\
   		\Sigma^{k + 1} =& \{r^{\frown}s: r \in \Sigma^{k}, s \in \Sigma\}\\
   		~\\
   		\Sigma^{\ast} =& \bigcup_{k \ge 0} \Sigma^{k}
   		\end{align*}
   	Onde escrevemos $\sigma_i^{\frown}\sigma_j$ para a concatenação entre as cadeias (\textit{strings}) $\sigma_i$ e $\sigma_j$ e $\square$ é uma cadeia vazia (a única cadeia de comprimento zero).
  
   	
   	\subsection{Fórmulas bem formadas}
   	
   	A linguagem em si é descrita pelo subconjunto $\mathscr{L} \subseteq \Sigma^{\ast}$, ou seja, está compreendida entre as possíveis combinações dos símbolos de seu alfabeto. Isto não é, contudo, suficiente para que a linguagem tenha o significado almejado. Vamos caracterizar $\mathscr{L}$ como sendo o conjunto das \textbf{Fórmulas bem formadas}. Uma Fórmula deste tipo é definida pelas afirmações que seguem. A definição se dá de maneira recursiva.
   	
   	\begin{itemize}
   		\item Constantes e variáveis são fórmulas atômicas. Toda fórmula atômica é bem formada.
   		
   		\item Se $\alpha$ é uma fórmula bem formada, $\neg \alpha$ e $(\alpha)$ também o são.
   		
   		\item Se $\alpha$ e $\beta$ são fórmulas bem formadas, $\alpha \vee \beta$, $\alpha \wedge \beta$ e $\alpha \rightarrow \beta$ também o são, assim como $\alpha \leftarrow \beta$ e $\alpha \leftrightarrow \beta$.
   		
   		\item Nada mais é uma fórmula bem formada.
   	\end{itemize}
   
	\section{Otimização}
	
	
	\begin{align*}
	\min_{x \in X} f(x)
	\end{align*}
	
	\subsection{Otimização Inteira}
	
    
    \chapter{Tipos}
    
    \section{Números}
    
    \section{Matrizes}
    
    \section{Variáveis}
    
    \chapter{Sintaxe do SATish}
    
    \begin{satish}
    ?epsilon: 1.5E-08;
    
    m = 3;
    n = 5;
    
    x[m] = {(1) : 1, (m) : -1};
    y[n] = {(1) : 0, (n) : +1};
    
    #{---------------------}
    * THIS IS SOME MULTI- *
    * LINE COMMENT...     *
    {---------------------}#
    
    # Integrity Constraints
    (int) A:
    @{i=[1:m]}        # forall i from 1 to m
    ${j=[1:n], j > i} # exists j from 1 to n for j greater than i
    
    x[i] -> y[j];
    
    # Optmality Constraints
    (opt) X[1]:
    @{i=[1:m]}         # forall i from 1 to m
    ${j=[1:n], i != j} # forall j from 1 to n where i is different from j
    
    x[i] & y[j];
    \end{satish}
    
    \section{Comentários}
    
    \section{Diretivas}
    
    \section{Atribuição}
    
    \section{Matrizes}
    
    \section{Definição de Restrições}
    
    \chapter{Exemplos}
    
    \section{Coloração de Grafos}
    
    \chapter*{Glossário}
    
    \begin{thebibliography}{20}
    	\bibitem{monteiro:10} MONTEIRO, B. F. \textbf{SATyrus2: Compilando Especificações de Racioncínio Lógico}. Dissertação (Engenharia de Sistemas e Computação) - PESC/COPPE, UFRJ. Rio de Janeiro, 2010.
    	
    	\bibitem{benevides:15} BENEVIDES, M. \textbf{Apostila de Lógica}. Rio de Janeiro, 2015.
    	
    	\bibitem{aho:86} AHO, Alfred e ULLMAN, Jeffrey, \textbf{Compiladores: Princípios, Técnicas e Ferramentas
    	Livro}, 1986. 
    \end{thebibliography}
    
\end{document}