\documentclass[brazil, MathSerif, aspectratio = 169]{beamer}


\usetheme{OsloMet}
\usepackage{homework}
\usepackage{style}
\usepackage{bm}

\author[Pedro]
{Pedro Maciel Xavier}
\title{III - Dualidade}
\subtitle{\small Otimização \textbullet{} Teoria dos Jogos \textbullet{} Lógica Proposicional}

\begin{document}
% Use
%
%     \begin{frame}[allowframebreaks]
%
% if the TOC does not fit one frame.
\begin{frame}%%[allowframebreaks]%%
    {Sumário}
    \tableofcontents
\end{frame}

%% Disable the logo in the lower right corner:
\hidelogo

\section{Tópicos Anteriores}
\begin{frame}{Tópicos Anteriores}

    Compreender melhor:
    \begin{enumerate}%%[<+-|alert@+>]
        \item 
    \end{enumerate}
\end{frame}

\section{\emph{Quadratic Unconstrained Binary Optimization}}

\subsection{Definição e características}

\begin{frame}{QUBO}
    \begin{definition}[\emph{Quadratic Unconstrained Binary Optimization}]
        Um problema de otimização é assim denominado se pode ser escrito na forma 
        \begin{align*}
            \text{minimizar } &f(\vet{x})\\
            \text{sujeito a } &\vet{x} \in \set{0, 1}^{n}
        \end{align*}
        onde $f(\vet{x}) = \vet{x}\T \vet{Q}\, \vet{x}$ para $\vet{Q} \in \R^{n \times n}$. Mais especificamente, $\vet{Q}$ é uma matriz simétrica ou triangular superior.
    \end{definition}    
\end{frame}

\begin{frame}{QUBO}
    \begin{observation}
        É muito importante, para a formulação, considerar a idempotência das variáveis binárias. Isto é, se $x \in \set{0, 1}$ então $x^2 = x$. Indutivamente, $x^n = x, n > 0$. Isso faz com que a diagonal principal da matriz $\vet{Q}$ represente os termos lineares.\par
        ~\\
        Deste fato vem também uma maneira de reduzir o grau das conexões. Como vimos anteriormente, qualquer termo de grau elevado mas com apenas um variável pode ser reduzido ao caso linear. De maneira análoga, termos de qualquer grau em duas variáveis pode ser trazido ao caso quadrático.\par
    \end{observation}
\end{frame}
%% ---------------------------------------------------------------------------

\subsection{Resolvendo}

\begin{frame}{Annealing}
    O \emph{Annealing} é um dos processos mais populares para solucionar o \emph{QUBO}. Dentre os principais métodos desta classe estão:
    \begin{itemize}%[<+-|structure@+>]
        \item \emph{Simulated Annealing}
        \item \emph{Quantum Annealing} (\emph{D-Wave})
        \item \emph{Digital Annealing} (\emph{Fujitsu})
    \end{itemize}
    \pause
    De um modo geral, é difícil encontrar soluções de qualidade em computadores convencionais. É um problema NP-Difícil.
\end{frame}

%% Enable the logo in the lower right corner:
\section{Satyrus}

\begin{frame}{Satyrus}

    No contexto do Satyrus, temos uma equação de energia a minimizar dada por
        \begin{align*}
            \mathbb{E} &= \mathbb{E}_\text{opt} + \mathbb{E}_\text{int}\\
                       &= \sum_{i} \mathcal{H} \left( \varphi_i \right) %%
                       + \sum_j \pmb{\lambda}_j \mathcal{H} \left( \neg \varphi_j \right)
        \end{align*}
    onde $\lambda_j$ é a penalidade associada à $j$-ésima restrição de integridade e $\mathcal{H} \left(\,\cdot\,\right)$ é o mapeamento. Portanto, é possível escrever um problema modelado pelo Satyrus como
        \begin{align*}
            \text{minimizar } &f(\mathbf{x}) + \pmb{\lambda} \cdot g(\mathbf{x})\\
            \text{sujeito a } &\mathbf{x} \in \{0, 1\}^{n}
        \end{align*}
\end{frame}

\begin{frame}{Satyrus}%%
    Resta saber se é possível escrever
        $f(\vet{x}) + \pmb{\lambda} \cdot g(\vet{x}) = \vet{x}\T \vet{Q}\, \vet{x}$.
    Como, por construção, tanto $f(\vet{x})$ quanto $g(\vet{x})$ são polinômios nas componentes de $\vet{x}$, isso pode ser feito ao aplicar uma redução dos termos com três ou mais variáveis.
    
    
\end{frame}

\section{Redução do grau}
\begin{frame}{Redução do grau}

    Compreender melhor:
    \begin{enumerate}%%[<+-|alert@+>]
        \item Redução do grau
        \item Topologia \emph{D-Wave}
    \end{enumerate}
\end{frame}

\section{Referências}
\begin{frame}%%[allowframebreaks]%%
    {Referências}

    \begin{thebibliography}{}

        % Article is the default.
        \setbeamertemplate{bibliography item}[article]

        \bibitem{glover:2016}
        Fred Glover, Gary Kochenberger, Yu Du
        \newblock \emph{Quantum Bridge Analytics I: A Tutorial on Formulating and Using QUBO Models.}
        % \newblock \emph{Optimization Letters}.
        % \newblock Springer-Verlag, 2011.

        % \setbeamertemplate{bibliography item}[book]

        % \bibitem{luenberger:2008}
        % David G. Luenberger, Yiniu Ye
        % \newblock \emph{Linear and Nonlinear Programming}
        % \newblock Springer-Verlag, 2008.


        \setbeamertemplate{bibliography item}[link]

        \bibitem{dwave:2016}
        D-Wave Systems
        \newblock \emph{Problem-Solving Handbook}
        \newblock \url{https://docs.dwavesys.com/docs/latest/c_handbook_3.html}

    \end{thebibliography}
\end{frame}

\end{document}
