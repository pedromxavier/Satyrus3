\documentclass[12pt]{satyrus}
\usepackage{satyrus}
\usepackage[brazil]{babel}

\title{Satyrus III}
\author{Pedro Maciel Xavier}

\begin{document}
    \maketitle
    
    \newpage

    \tableofcontents

    \chapter{Introdução}
    
    - SATyrus é uma plataforma\\
    - SATish é a linguagem\\
    
    \section{Motivação}

    \section{Ficha Técnica}

    \section{Implementação}

    \section{Uso}
    
    \subsection{Instalação}
    
    A instalação pode demandar privilégios de administrador.
    
	\begin{bash}
	~$ git clone
	~$ cd Satyrus3
	~$ sudo python3 setup.py install
	\end{bash}
	
	\begin{shell}
	C:\Users\User> git clone
	C:\Users\User> cd Satyrus3
	C:\Users\User\Satyrus3> python setup.py install
	\end{shell}

    \subsection{Execução}
    
    Escreva seu código em um arquivo de extensão \code{.sat}.
    
	\begin{bash}
	~$ satyrus script.sat
	\end{bash}
	
	\begin{shell}
	C:\Users\User> python -m satyrus script.sat
	\end{shell}

    \chapter{Conceitos Teóricos}

	\section{Compiladores}
	
	

    \section{Lógica Proposicional}
    
   	A Lógica Proposicional é um sistema formal
   	
   	\subsection{Alfabeto}
   	
   	\subsection{Fórmulas bem formadas}
   	
   	\begin{enumerate}
   		\item Toda fórmula atômica é bem formada.
   		
   		\item Se $\alpha$ é uma fórmula bem formada, $\neg \alpha$ também o é.
   		
   		\item Se $\alpha$ e $\beta$ são fórmulas bem formadas, $\alpha \vee \beta$, $\alpha \wedge \beta$ e $\alpha \to \beta$ também o são.
   		
   		\item Nada mais é uma fórmula bem formada.
   	\end{enumerate}
   
	\section{Otimização}
    
    \chapter{Tipos}
    
    \section{Números}
    
    \section{Matrizes}
    
    \section{Variáveis}
    
    \chapter{Sintaxe do SATish}
    
    \section{Comentários}
    
    \section{Diretivas}
    
    \section{Atribuição}
    
    \section{Matrizes}
    
    \section{Definição de Restrições}
    
    \chapter{Exemplos}
    
    \section{Coloração de Grafos}
    
    \chapter*{Glossário}
    
    \begin{thebibliography}{20}
    	\bibitem{monteiro:10} MONTEIRO, B. F. \textbf{SATyrus2: Compilando Especificações de Racioncínio Lógico}. Dissertação (Engenharia de Sistemas e Computação) - PESC/COPPE, UFRJ. Rio de Janeiro, 2010.
    	
    	\bibitem{benevides:15} BENEVIDES, M. \textbf{Apostila de Lógica}. Rio de Janeiro, 2015.
    \end{thebibliography}
    
\end{document}